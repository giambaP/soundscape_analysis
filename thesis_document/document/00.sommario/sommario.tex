\begin{abstract}
Il seguente studio si colloca nell’ambito della \textit{Pattern Recognition} e del \textit{Machine Learning}, in
particolare alla sua applicazione nell’analisi di \textit{soundscape}. Per \textit{soundscape} si intende
l’ambiente sonoro composto da suoni naturali e artificiali, relativi ad uno specifico luogo
geografico.

L’analisi si propone di caratterizzare \textit{soundscapes} mediante tecniche di classificazione, i.e.
metodi in grado di assegnare un oggetto ad un insieme predefinito di categorie.

E’ stata implementata una \textit{pipeline} di \textit{Pattern Recognition}. Il segnale audio è stato
caratterizzato attraverso classiche tecniche d’estrazione delle \textit{features} - o caratteristiche (ad
esempio \textit{features} legate al contenuto spettrale, alla forma, o al timbro). Mediante tecniche di
classificazione si è provato a misurare la capacità discriminativa di queste caratteristiche per
classificare diverse categorie, come ad esempio il giorno dalla notte oppure la presenza o
assenza di un temporale.

Sulla base di questo studio, in una fase successiva si è effettuata un'analisi esplorativa
mediante algoritmi di \textit{anomaly detection} (algoritmi in grado di evidenziare pattern anomali)
per inferire informazioni e analizzare eventuali singolari pattern emersi.

La \textit{pipeline} proposta è stata testata su dati raccolti in ambito di un monitoraggio acustico
nella \textit{Riserva Naturale Los Yátaros}, nel dipartimento di \textit{Boyacá} in \textit{Colombia}. La riserva
presenta una biodiversità acustica molto particolare.

I risultati ottenuti sono incoraggianti e il loro contributo potrà migliorare la conoscenza nello
studio acustico degli ecosistemi.

\end{abstract}