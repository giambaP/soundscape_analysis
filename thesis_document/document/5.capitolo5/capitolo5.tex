\chapter{Anomaly Detection}
In questo capitolo si andrà ad esporre la terza fase dello studio, che si occupa di studiare
algoritmi di A.D. per individuare eventuali suoni anomali. Nel contesto dei \textit{soundscape} gioca
un ruolo molto interessante, per identificare suoni o eventi insoliti che potrebbero indicare
cambiamenti ambientali, rumori estranei o fenomeni singolari: per esempio, la presenza di un
cacciatore, oppure un animale non appartenente alla fauna del luogo. In questa analisi si è
deciso di sperimentare l’utilizzo di tale metodologia applicando le conoscenze ottenute dalle
fasi precedenti, per inferire informazioni ed esaminare quali elementi vengono classificati
come anomali.

\section{Configurazione \textit{features} e metodi}
Nel seguente paragrafo saranno descritte le \textit{features} utilizzate in questo studio e le
configurazioni dei metodi esposti nel paragrafo 2.3. Come menzionato sopra, la conoscenza
ottenuta con lo studio di classificazione ci ha permesso di individuare \textit{feature} che riescono ad
caratterizzare chiaramente il \textit{dataset}. Inoltre, dai risultati si è potuto verificare quale delle due
finestre sia in grado di definire le caratteristiche più rilevanti. Si è quindi utilizzato il set di
feature della versione FS1 di entrambe le forme di dati, per un totale di 38 \textit{features}, 19
normali (FS1.NOR) e 19 standardizzate (FS1.STD).

Per quanto riguarda i metodi, le versioni utilizzate presentano delle configurazioni
considerabili come standard, quindi non sono state fatte particolari ottimizzazioni. IF utilizza
100 alberi decisionali, un valore largamente utilizzato; la scelta di un numero maggiore
potrebbe determinare una maggiore robustezza del modello, ma incrementa notevolmente
anche il tempo di calcolo. Il numero di campionamenti presi per ogni albero è di 256
elementi. Per LOF, invece, si è fissato per k (vicini) il valore di venti elementi, e come
metrica di distanza, quella euclidea, già descritta precedentemente. Per l’algoritmo OCSVM,
infine, si è utilizzato il kernel gaussiano. Il fattore di contaminazione, ovvero la soglia
decisionale per definire quali elementi trattare come anomalie, è stato impostato a zero per
tutti e tre gli algoritmi. Con tale valore il sistema non avendo una soglia definita per capire se
un elemento è un'anomalia o meno, tratta tutti gli elementi come normali, assegnandogli solo
un punteggio di anormalità. Si è deciso di non considerare il fattore di contaminazione dato
che, nell’ambito dello studio è molto difficile definire una soglia a priori che possa definire l'anormalità. 
Si è deciso di valutare invece gli elementi che con maggior frequenza risultano tra i punteggi più alti dei risultati delle
configurazioni di \textit{features}.

\section{Validazione dei risultati}
In questo paragrafo verrà descritta la modalità di validazione per le tecniche A.D. nel
contesto dello studio. Gli algoritmi sopra descritti fanno parte della tipologia di approcci non
supervisionati, ovvero non necessitano quindi di etichette per poter funzionare. Per avere la
validazione, è stato comunque impiegato il dataset DATA2, che è etichettato, così da potersi
servire delle conoscenze sul contenuto per verificare il risultato.

Le modalità di analisi prevedono trenta esecuzioni dell’algoritmo, calcolando la media dei
punteggi di anomalia, in maniera tale da essere indipendenti dal valore della singola
esecuzione. L’approccio viene replicato per ciascuna \textit{feature} (38, 19 normali e 19
standardizzate), considerando poi solo i tre risultati con punteggio maggiore per feature,
ottenendo un insieme totale di 38 x 3 risultati per ogni metodo. Per la valutazione del metodo
sono state estratte le prime cinque occorrenze più numerose dai risultati del metodo. Per avere visione complessiva,
invece, sono state estratte le prime cinque occorrenze più numerose considerando tutti e tre
gli insiemi di risultati.

\section{Risultati}
Dopo aver esposto la modalità di applicazione dell’analisi si andrà a descrivere i risultati
ottenuti. Nella tabella in figura 5.1 è possibile avere una visione dei risultati come descritti nel
paragrafo precedente. I dati sono divisi in quattro gruppi definiti dalla prima colonna: i primi
tre riguardano i risultati dei singoli metodi applicati ovvero IF, LOF e O, e il quarto i risultati
sul totale. Per ognuno gruppo vengono esposti i primi cinque oggetti che hanno ottenuto più
occorrenze. Nella seconda colonna si ha la descrizione dell’oggetto riscontrato come
anomalia e nella terza colonna il numero di occorrenze rilevate all'interno del gruppo che identificano il file come anomalo. Nelle restanti colonne sono elencati i suoni presenti all’interno dell’audio riscontrato come
anomalia. Per ogni gruppo viene indicato il relativo risultato percentuale della presenza sonora di un determinato suono all'interno del gruppo.

Dal punto di vista dei singoli suoni, gli oggetti più anomali presentano: il rumore dei veicoli
rilevato per un 40\% su IF e O, per un 60\% invece per L; il suono degli uccelli U che è sempre
presente per un 60\% e con una distribuzione simile in tutte e tre le casistiche; il suono dei
grilli G, che è emerso presente per un 40\% e solo in L; il suono del fiume/cascata C, che,
come prevedibile, appare molto diffuso tra i risultati, ma si può ipotizzare che non abbia nulla
di particolare e che la sua presenza sia dovuta unicamente alla sua ampia distribuzione nei
dati; il suono della pioggia P che si è notevolmente osservato, per un 80\%, e si può affermare
che valgono le stesse conclusione appena fatte per il caso precedente; infine, le interferenze e
i suoni sconosciuti non sono stati rilevati come anomalie, contrariamente alle aspettative. Ci
si poteva aspettare che sarebbero emersi data la loro minima presenza e particolarità, ma
invece sono stati rilevati solo per un caso, e solo dal metodo O.

Come è possibile notare, dalla prospettiva dei metodi, si vede che IF e O hanno espresso delle
distribuzioni simili negli oggetti anomali rilevati nella BIO con il suono degli uccelli U per un
60\% e nessun caso che contenesse dei grilli G, nella GEO con il suono di cascata C e pioggia
P per un 80\% e infine nessuna presenza di interferenza. Invece, il metodo L ha dimostrato
una sensibilità diversa, maggiore per il suono della cascata C, rilevata nel 100\% degli oggetti
anomali, e una minore invece per la pioggia P per un 40\%.

Invece, se si osserva dal punto di vista degli insiemi di suoni ANT/BIO/GEO, si può notare
come la GEO sia molto presente, per almeno un 80\%. Per una parte si possono trarre le stesse
conclusioni definite prima sulla distribuzione dei suoni C e P, ma per quanto riguarda quello
di T, si può ipotizzare che non lo sia. Infatti, la sua distribuzione nel dataset di riferimento è
di circa del 60\%, e la sua bilanciata presenza tra i risultati si può ritenere molto interessante.
Similmente, per ANT e BIO è possibile sostenere l’ipotesi che il risultato ottenuto sia
rilevante: per il primo sulla bassa numerosità dei dati in input, per il secondo invece, data la
sua grande distribuzione, perché non si è propagato similmente a quanto osservato nei suoni
più rilevati della GEO.

Infine, da una prospettiva temporale, si manifesta che l’80\% dei risultati sul totale, ma anche
nei singoli algoritmi, si concentra alle ore 14:00, nella seconda parte della giornata. Lo stesso
per l’elemento identificato maggiormente come anomalia che risulta invece verso le ore
18:00.

%\begin{table}[ht]
%	\centering
%	\begin{tabular}{@{}lclcccccccc@{}}
%		\toprule
%		& \multicolumn{1}{l}{} &  & \multicolumn{1}{l}{\cellcolor[HTML]{FCEFEE}\textbf{ANT.}} & \multicolumn{2}{c}{\cellcolor[HTML]{9FE887}{\color[HTML]{000000} \textbf{BIO.}}} & \multicolumn{3}{c}{\cellcolor[HTML]{B0E4F3}\textbf{GEO.}} & \multicolumn{2}{c}{\cellcolor[HTML]{FFFFFF}\textbf{ALTRO}} \\ \cmidrule(l){4-11} 
%		\textbf{Gruppo} & \multicolumn{1}{c}{\textbf{\# occ.}} & \multicolumn{1}{c}{\textbf{Audio}} & \textbf{V} & \textbf{U} & \textbf{G} & \textbf{C} & \textbf{P} & \textbf{F} & \textbf{I} & \textbf{S} \\ \midrule
%		& 22 & 20200323\_180000.WAV & x & x &  & x & x &  &  &  \\
%		& 18 & 20200303\_140000.WAV &  &  &  &  & x & x &  &  \\
%		& 12 & 20200304\_140000.WAV &  &  &  & x & x & x &  &  \\
%		& 8 & 20200309\_060000.WAV & x & x &  & x &  &  &  &  \\
%		& 8 & 20200316\_220000.WAV &  & x &  & x & x &  &  &  \\ \cmidrule(l){2-11} 
%		\multirow{-6}{*}{\textbf{IF}} & \multicolumn{2}{c}{\cellcolor[HTML]{EFEFEF}\textbf{RISULTATO \%}} & \cellcolor[HTML]{EFEFEF}\textbf{40} & \cellcolor[HTML]{EFEFEF}\textbf{60} & \cellcolor[HTML]{EFEFEF}\textbf{0} & \cellcolor[HTML]{EFEFEF}\textbf{80} & \cellcolor[HTML]{EFEFEF}\textbf{80} & \cellcolor[HTML]{EFEFEF}\textbf{40} & \cellcolor[HTML]{EFEFEF}\textbf{0} & \cellcolor[HTML]{EFEFEF}\textbf{0} \\ \midrule
%		& 15 & 20200311\_140000.WAV &  &  &  & x & x & x &  &  \\
%		& 15 & 20200323\_180000.WAV & x & x &  & x & x &  &  &  \\
%		& 10 & 20200318\_140000.WAV & x & x &  & x &  &  &  &  \\
%		& 9 & 20200301\_180000.WAV &  & x & x & x &  & x &  &  \\
%		& 8 & 20200325\_020000.WAV & x &  & x & x &  &  &  &  \\ \cmidrule(l){2-11} 
%		\multirow{-6}{*}{\textbf{LOF}} & \multicolumn{2}{c}{\cellcolor[HTML]{EFEFEF}\textbf{RISULTATO \%}} & \cellcolor[HTML]{EFEFEF}{\color[HTML]{000000} \textbf{60}} & \cellcolor[HTML]{EFEFEF}{\color[HTML]{000000} \textbf{60}} & \cellcolor[HTML]{EFEFEF}{\color[HTML]{000000} \textbf{40}} & \cellcolor[HTML]{EFEFEF}{\color[HTML]{000000} \textbf{100}} & \cellcolor[HTML]{EFEFEF}{\color[HTML]{000000} \textbf{40}} & \cellcolor[HTML]{EFEFEF}{\color[HTML]{000000} \textbf{40}} & \cellcolor[HTML]{EFEFEF}{\color[HTML]{000000} \textbf{0}} & \cellcolor[HTML]{EFEFEF}{\color[HTML]{000000} \textbf{0}} \\ \midrule
%		& 6 & 20200323\_180000.WAV & x & x &  & x & x &  &  &  \\
%		& 5 & 20200309\_060000.WAV & x & x &  & x &  &  &  &  \\
%		& 4 & 20200314\_180000.WAV & x & x &  & x & x & x &  & x \\
%		& 4 & 20200328\_140000.WAV &  &  &  & x & x & x &  &  \\
%		& 3 & 20200303\_140000.WAV &  &  &  &  & x & x &  &  \\ \cmidrule(l){2-11} 
%		\multirow{-6}{*}{\textbf{OCSVM}} & \multicolumn{2}{c}{\cellcolor[HTML]{EFEFEF}\textbf{RISULTATO \%}} & \cellcolor[HTML]{EFEFEF}\textbf{60} & \cellcolor[HTML]{EFEFEF}\textbf{60} & \cellcolor[HTML]{EFEFEF}\textbf{0} & \cellcolor[HTML]{EFEFEF}\textbf{80} & \cellcolor[HTML]{EFEFEF}\textbf{80} & \cellcolor[HTML]{EFEFEF}\textbf{80} & \cellcolor[HTML]{EFEFEF}\textbf{0} & \cellcolor[HTML]{EFEFEF}\textbf{20} \\ \midrule
%		& 43 & 20200323\_180000.WAV & x & x &  &  & x &  &  &  \\
%		& 25 & 20200303\_140000.WAV &  &  &  &  & x & x &  &  \\
%		& 21 & 20200304\_140000.WAV &  &  &  & x & x & x &  &  \\
%		& 18 & 20200311\_140000.WAV &  &  &  & x & x & x &  &  \\
%		\multirow{-5}{*}{\textbf{TOTAL}} & 14 & 20200318\_140000.WAV & x & x &  & x &  &  &  &  \\ \midrule
%		& \multicolumn{2}{c}{\cellcolor[HTML]{EFEFEF}\textbf{RISULTATO \%}} & \cellcolor[HTML]{EFEFEF}\textbf{40} & \cellcolor[HTML]{EFEFEF}\textbf{40} & \cellcolor[HTML]{EFEFEF}\textbf{0} & \cellcolor[HTML]{EFEFEF}\textbf{80} & \cellcolor[HTML]{EFEFEF}\textbf{80} & \cellcolor[HTML]{EFEFEF}\textbf{60} & \cellcolor[HTML]{EFEFEF}\textbf{0} & \cellcolor[HTML]{EFEFEF}\textbf{0} \\ \bottomrule
%	\end{tabular}
%	\caption{Risultati dell'\textit{anomaly detection} suddivisi dalla colonna gruppo i risultati dei tre metodi IF, LOF e OCSM, e per ultimo il risultato sull'insieme totale. La seconda colonna descrive il numero di occorrenzein ogni gruppo per il relativo audio, specificato nella terza colonna. Il nome dell'audio è composto dalla data (in formato anno con 4 cifre, il mese con 2 cifre e il giorno con 2 cifre), un separatore '\textit{\_}', l'ora (in formato ora 2 cifre e minuti 2 cifre) e infine l'estensione \textit{.WAV}.
%		Per ogni audio nelle colonne successive sono indicati mediante una \textit{x} le classi di suoni che gli appartengono dei relativi insieme ANT/BIO/GEO e ALTRO per le interferenze I e i suoni sconosciuti S. }
%	\label{fig:51}
%\end{table}

\begin{table}[ht]
	\centering
	\begin{tabular}{@{}llccccccccc@{}}
		\toprule
		& \multicolumn{1}{l}{} &  & \multicolumn{1}{l}{\cellcolor[HTML]{FCEFEE}\textbf{ANT.}} & \multicolumn{2}{c}{\cellcolor[HTML]{9FE887}{\color[HTML]{000000} \textbf{BIO.}}} & \multicolumn{3}{c}{\cellcolor[HTML]{B0E4F3}\textbf{GEO.}} & \multicolumn{2}{c}{\cellcolor[HTML]{FFFFFF}\textbf{ALTRO}} \\ \cmidrule(l){4-11} 
		\textbf{Gruppo} & \multicolumn{1}{c}{\textbf{Audio}} & \multicolumn{1}{c}{\textbf{\# occ.}} & \textbf{V} & \textbf{U} & \textbf{G} & \textbf{C} & \textbf{P} & \textbf{F} & \textbf{I} & \textbf{S} \\ \midrule
		& 20200323\_180000.WAV & 22 & x & x &  & x & x &  &  &  \\
		& 20200303\_140000.WAV & 18 &  &  &  &  & x & x &  &  \\
		& 20200304\_140000.WAV & 12 &  &  &  & x & x & x &  &  \\
		& 20200309\_060000.WAV & 8 & x & x &  & x &  &  &  &  \\
		& 20200316\_220000.WAV & 8 &  & x &  & x & x &  &  &  \\ \cmidrule(l){2-11} 
		\multirow{-6}{*}{\textbf{IF}} & \multicolumn{2}{c}{\cellcolor[HTML]{EFEFEF}\textbf{RISULTATO \%}} & \multicolumn{1}{c}{\cellcolor[HTML]{EFEFEF}\textbf{40}} & \multicolumn{1}{c}{\cellcolor[HTML]{EFEFEF}\textbf{60}} & \multicolumn{1}{c}{\cellcolor[HTML]{EFEFEF}\textbf{0}} & \multicolumn{1}{c}{\cellcolor[HTML]{EFEFEF}\textbf{80}} & \multicolumn{1}{c}{\cellcolor[HTML]{EFEFEF}\textbf{80}} & \multicolumn{1}{c}{\cellcolor[HTML]{EFEFEF}\textbf{40}} & \multicolumn{1}{c}{\cellcolor[HTML]{EFEFEF}\textbf{0}} & \multicolumn{1}{r}{\cellcolor[HTML]{EFEFEF}\textbf{0}} \\ \midrule
		& 20200311\_140000.WAV & 15 &  &  &  & x & x & x &  &  \\
		& 20200323\_180000.WAV & 15 & x & x &  & x & x &  &  &  \\
		& 20200318\_140000.WAV & 10 & x & x &  & x &  &  &  &  \\
		& 20200301\_180000.WAV & 9 &  & x & x & x &  & x &  &  \\
		& 20200325\_020000.WAV & 8 & x &  & x & x &  &  &  &  \\ \cmidrule(l){2-11} 
		\multirow{-6}{*}{\textbf{LOF}} & \multicolumn{2}{c}{\cellcolor[HTML]{EFEFEF}\textbf{RISULTATO \%}} & \multicolumn{1}{c}{\cellcolor[HTML]{EFEFEF}{\color[HTML]{000000} \textbf{60}}} & \multicolumn{1}{c}{\cellcolor[HTML]{EFEFEF}{\color[HTML]{000000} \textbf{60}}} & \multicolumn{1}{c}{\cellcolor[HTML]{EFEFEF}{\color[HTML]{000000} \textbf{40}}} & \multicolumn{1}{c}{\cellcolor[HTML]{EFEFEF}{\color[HTML]{000000} \textbf{100}}} & \multicolumn{1}{c}{\cellcolor[HTML]{EFEFEF}{\color[HTML]{000000} \textbf{40}}} & \multicolumn{1}{c}{\cellcolor[HTML]{EFEFEF}{\color[HTML]{000000} \textbf{40}}} & \multicolumn{1}{c}{\cellcolor[HTML]{EFEFEF}{\color[HTML]{000000} \textbf{0}}} & \multicolumn{1}{c}{\cellcolor[HTML]{EFEFEF}{\color[HTML]{000000} \textbf{0}}} \\ \midrule
		& 20200323\_180000.WAV & 6 & x & x &  & x & x &  &  &  \\
		& 20200309\_060000.WAV & 5 & x & x &  & x &  &  &  &  \\
		& 20200314\_180000.WAV & 4 & x & x &  & x & x & x &  & x \\
		& 20200328\_140000.WAV & 4 &  &  &  & x & x & x &  &  \\
		& 20200303\_140000.WAV & 3 &  &  &  &  & x & x &  &  \\ \cmidrule(l){2-11} 
		\multirow{-6}{*}{\textbf{OCSVM}} & \multicolumn{2}{c}{\cellcolor[HTML]{EFEFEF}\textbf{RISULTATO \%}} & \multicolumn{1}{c}{\cellcolor[HTML]{EFEFEF}\textbf{60}} & \multicolumn{1}{c}{\cellcolor[HTML]{EFEFEF}\textbf{60}} & \multicolumn{1}{c}{\cellcolor[HTML]{EFEFEF}\textbf{0}} & \multicolumn{1}{c}{\cellcolor[HTML]{EFEFEF}\textbf{80}} & \multicolumn{1}{c}{\cellcolor[HTML]{EFEFEF}\textbf{80}} & \multicolumn{1}{c}{\cellcolor[HTML]{EFEFEF}\textbf{80}} & \multicolumn{1}{c}{\cellcolor[HTML]{EFEFEF}\textbf{0}} & \multicolumn{1}{c}{\cellcolor[HTML]{EFEFEF}\textbf{20}} \\ \midrule
		& 20200323\_180000.WAV & 43 & x & x &  &  & x &  &  &  \\
		& 20200303\_140000.WAV & 25 & &  &  &  & x & x &  &  \\
		& 20200304\_140000.WAV & 21 &  &  &  & x & x & x &  &  \\
		& 20200311\_140000.WAV & 18 &  &  &  & x & x & x &  &  \\
		\multirow{-5}{*}{\textbf{TOTAL}} & 20200318\_140000.WAV & 14 & x & x &  & x &  &  &  &  \\ \midrule
		& \multicolumn{2}{c}{\cellcolor[HTML]{EFEFEF}\textbf{RISULTATO \%}} & \multicolumn{1}{c}{\cellcolor[HTML]{EFEFEF}\textbf{40}} & \multicolumn{1}{c}{\cellcolor[HTML]{EFEFEF}\textbf{40}} & \multicolumn{1}{c}{\cellcolor[HTML]{EFEFEF}\textbf{0}} & \multicolumn{1}{c}{\cellcolor[HTML]{EFEFEF}\textbf{80}} & \multicolumn{1}{c}{\cellcolor[HTML]{EFEFEF}\textbf{80}} & \multicolumn{1}{c}{\cellcolor[HTML]{EFEFEF}\textbf{60}} & \multicolumn{1}{r}{\cellcolor[HTML]{EFEFEF}\textbf{0}} & \multicolumn{1}{r}{\cellcolor[HTML]{EFEFEF}\textbf{0}} \\ \bottomrule
	\end{tabular}
	\caption{Risultati dell'\textit{anomaly detection}. Suddivisi dalla colonna gruppo i risultati dei tre metodi IF, LOF e OCSM, e per ultimo il risultato sull'insieme totale. La seconda colonna descrive il file audio rilevato come anomalia e per ognuno, nella terza colonna, viene specificato il numero di occorrenze rilevate nel gruppo. Il nome dell'audio è composto dalla data (in formato anno con 4 cifre, il mese con 2 cifre e il giorno con 2 cifre), un separatore '\textit{\_}', l'ora (in formato ora 2 cifre e minuti 2 cifre) e infine l'estensione \textit{.WAV}.  Per ogni audio nelle colonne successive sono indicati mediante una \textit{x} le classi di suoni che gli appartengono dei relativi insieme ANT/BIO/GEO e ALTRO per le interferenze I e i suoni sconosciuti S. }
	\label{fig:51}
\end{table}