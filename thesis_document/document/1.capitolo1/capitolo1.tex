\chapter{Introduzione}
In questo capitolo saranno introdotte le nozioni fondamentali per comprendere l’ambito su
cui si è sviluppato questo studio. Saranno definiti i concetti relativi a \textit{Pattern Recognition} e
\textit{Machine Learning}, al loro utilizzo nell’analisi di \textit{soundscape}, e in generale nell’analisi di
audio nel campo dell’ecologia. Nel paragrafo successivo, si andrà ad esplicitare il fine che ha
suggerito lo sviluppo di questo studio.

% \\ forza a capo manuale
\section{\textit{Pattern Recognition} e \textit{Machine Learning} \\ per l'analisi di soundscapes e in generale di audio per l'ecologia}
\textit{Pattern recognition} (PR) e \textit{Machine learning} (ML) [1] rappresentano una branca
fondamentale dell’intelligenza artificiale, in particolare un insieme di tecniche utilizzate per
estrarre informazioni dai dati tramite il riconoscimento automatico di specifici schemi,
definiti \textit{pattern}. In modo approssimato, si può dire che sono equivalenti, poiché condividono
obiettivi, strumenti e approcci.

Il loro impiego è noto in molteplici ambiti: dal riconoscimento vocale o di immagini,
all’elaborazione del linguaggio naturale, dai sistemi di raccomandazione, al monitoraggio in
tempo reale e molti altri [1]. Tra questi emerge un contesto poco analizzato, che in letteratura
si presenta come una sfida ancora aperta: l’analisi di soundscape. Prima di esaminare nel
dettaglio come i \textit{soundscape} sono stati affrontati nella PR/ML, per chiarezza, si desidera
spiegare cosa si intende con tale definizione, le motivazioni per cui merita attenzione e le
varie problematiche annesse.

Quinn \textit{et al.} identificano i \textit{soundscape} come una “particolare combinazione di suoni in un
paesaggio” considerandola come “una caratterizzazione ecologica dei paesaggi” [2]. Gli
autori ritengono che la composizione di un soundscape si divide in quattro elementi
principali: l’antropofonia (ANT: indica l’attività antropogenica), la biofonia (BIO: intesa
come le vocalizzazioni della fauna selvatica), la geofonia (GEO: descrive i suoni dei
fenomeni meteorologici) [2] e infine la quiete (indicata come il suono dell’ambiente).

A tal proposito, è molto interessante la caratterizzazione fornita da Farina \textit{et al. }[3].
L’argomentazione descritta espone una visione alternativa più mirata e strutturata: separa il
concetto di \textit{sonoscape} da \textit{soundscape}. Con \textit{sonoscape} intende “il mosaico di tutte le non
interpretate informazioni sonore all’interno di un landscape” [3]. Da questa definizione si
deduce per esclusione l’interpretazione che l’autore attribuisce al \textit{soundscape}, ossia “un
\textit{sonoscape} che è stato cognitivamente interpretato in un mosaico di categorie di ANT, BIO e
GEO semioticamente interpretate da un organismo” [3]. Un'ulteriore suddivisione separa gli
elementi in quelle che definisce unità sonore, i \textit{sonotope} per gli \textit{sonoscape}, e i \textit{soundtope} per i
\textit{soundscape}. I primi vengono definiti dall’autore come una \textit{patch} spazialmente unica di suoni
non interpretati, mentre i secondi come suoni di ANT, BIO e GEO semioticamente
interpretati da un organismo [3]. Rispetto ad un’umana suddivisione in ANT/BIO/GEO [3],
questi concetti appena espressi consentirebbero una classificazione con maggiore dettaglio e
specificità. Ciononostante, i termini \textit{sonotope} e \textit{soundtope} sono tuttora relegati a mere
speculazioni a causa di una scarsità di evidenze empiriche [3].

Sebbene la definizione di \textit{soundscape} possa risultare complessa, molto chiara è invece la sua
importanza. Il ruolo che ricopre nell’ambiente naturale rappresenta un segnale della salute
dell’ecosistema. Tale segnale può essere utilizzato per studi ecologici [3], diviene un
significativo approfondimento della biodiversità e dell’impatto umano [2], può evidenziare
cambiamenti negli habitat dove la qualità acustica è fondamentale per la dimensione vitale e
il rumore umano risulta deleterio sulla biodiversità [2].

I vantaggi appena descritti supportano e incoraggiano l’analisi degli \textit{soundscape}. PR/ML
possono dare un grande contributo in tale processo. Sviluppare sistemi automatici mediante
tecniche di PR/ML permetterebbe di supportare le sfide riguardanti l’analisi dei dati e il
monitoraggio in tempo reale dell’ecosistema. La classificazione di \textit{soundscape} consente
l’identificazione automatica di suoni indesiderati su grandi quantità di dati, inoltre permette
di modellare gli effetti e le interazioni di suoni diversi, e utilizzare poi tali modelli per
identificare pattern spazio-temporali nell’attività sonora [2]. Introdurre un sistema euristico in
grado di monitorare la presenza o l’abbondanza di particolari specie potrebbe aiutare la
gestione di tale specie in una determinata zona, o addirittura evitarne l’estinzione. Allo stesso
modo, può essere utile per prevenire situazioni di pericolo come il bracconaggio. I dati
ricavati sarebbero fonte di studio per molti comportamenti animali in specifici periodi
dell’anno, come il corteggiamento.

Questa innovazione nell’analisi degli \textit{soundscape} non è priva di problematiche. Pochi studi di
ecoacustica hanno provato a classificare soundscapes utilizzando intere categorie sonore
come ANT/BIO/GEO e quiete [2]. Tale difficoltà si sviluppa su due elementi. In primo piano,
l’identificazione manuale delle sorgenti sonore è altamente dispendiosa in termini di tempo
[2]. Ciò è dovuto all’enorme quantità di dati da visionare manualmente che servono a censire
un \textit{dataset} di addestramento per i sistemi di PR/ML. Tanto più il \textit{dataset} risulta ampio e
dettagliato, maggiore sarà la qualità del sistema sviluppato. Oltre al tempo impiegato si
possono sottintendere anche i costi di tale opera. Il secondo punto riguarda le competenze
specifiche del settore. Infatti, per censire i dati di addestramento è richiesta una conoscenza
della vocalizzazione degli animali del contesto, determinando per necessità la scelta di
sviluppare \textit{dataset} di addestramento di piccole dimensioni [2]. Tali \textit{dataset} non riescono a
spiegare nel complesso il problema, limitando così la qualità dei sistemi realizzabili.

\section{Obiettivo della tesi: classificazione e analisi preliminare di \textit{anomaly detection}}
Il seguente studio si propone di studiare approcci di PR/ML per la caratterizzazione di un
soundscape.

E’ stata implementata una \textit{pipeline} di \textit{Pattern Recognition}. Il segnale audio è stato
caratterizzato attraverso classiche tecniche d’estrazione delle \textit{features} - o caratteristiche (ad
esempio features legate al contenuto spettrale, alla forma, o al timbro). Mediante tecniche di
classificazione si è provato a misurare la capacità discriminativa di queste caratteristiche per
classificare diverse categorie, come ad esempio il giorno dalla notte oppure la presenza o
assenza di un temporale.

Sulla base di questo studio, in una fase successiva si è effettuata un'analisi esplorativa
mediante algoritmi di \textit{anomaly detection }(algoritmi in grado di evidenziare pattern anomali)
per inferire informazioni e analizzare eventuali singolari pattern emersi.
