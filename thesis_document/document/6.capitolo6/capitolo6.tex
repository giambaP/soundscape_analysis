\chapter{Conclusione}
In questo studio sono state applicate tecniche di \textit{Pattern Recognition} e \textit{Machine Learning} per
caratterizzare mediante classificazione e \textit{anomaly detection} un \textit{soundscape}.

Come è già noto nella letteratura sui \textit{soundscape}, il contesto dello studio presenta notevoli
difficoltà intrinseche. Si deve tenere conto che lo stesso cervello umano, nonostante le sue
straordinarie capacità, fatica a distinguere e classificare determinati suoni.

Nell’analisi si è realizzato un sistema di classi e sono state studiate diverse \textit{features} per
caratterizzare un \textit{soundscape} ottenuto da uno studio di monitoraggio passivo della Riserva
Naturale Los Yátaros in Colombia. Si è proceduto in due fasi di classificazione mediante
categorie oggettive (per esempio giorno/notte) nella prima e categorie semantiche nella seconda. Infine,
sono stati è applicati algoritmi di \textit{anomaly detection} per individuare eventuali suoni anomali.

Nella classificazione, in entrambe le fasi si è ottenuto che la \textit{feature} migliore consiste nella
concatenazione delle medie dei gruppi di tipologie di \textit{features}, nella versione standardizzata e
in combinazione con la dimensione della finestra più ampia. Tra i problemi con categorie
oggettive la discriminazione sul luogo ha ottenuto risultati nettamente superiori rispetto al
resto. Nelle casistiche con categorie semantiche i problemi binari con due classi distinte
occupano la prima posizione distaccandosi di poco da quelli binari che discriminano la
presenza/assenza della classe. I problemi a tre classi, essendo un contesto multi etichetta
come nei precedenti casi ma dovendo discriminare più classi, hanno ottenuto risultati accettabili.
Nel particolare caso dei binari le \textit{feature} spettrali hanno ottenuto risultati interessanti, per i
ternari lo sono state quelle tonali.

L’\textit{anomaly detection} non ha rilevato particolari elementi significativi e non è emerso nessuno
schema particolare tra i dati. I suoni relativi a interferenze e a quelli sconosciuti non sono
emersi come anomalie. I suoni relativi alla geofonia sono stati riscontrati frequentemente. Si
è potuto notare che la presenza di dati anomali ricade nelle fasce orarie pomeridiane.
Il tentativo di filtraggio sperimentato non ha portato nessun vantaggio dal punto di vista della
qualità dell’audio, ma i risultati ottenuti sono comunque validi. La stessa combinazione di
caratteristiche vincenti emerse dalla classificazione si sono mantenute tali anche rimuovendo
parte delle frequenze.

Noti i vantaggi derivati dalla comprensione dei \textit{soundscape}, si termina questo studio con la
speranza che la ricerca possa avanzare e trovare soluzioni migliori per approfondire la
conoscenza del nostro ecosistema.

